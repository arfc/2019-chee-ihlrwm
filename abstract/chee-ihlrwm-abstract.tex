\documentclass{anstrans}
%%%%%%%%%%%%%%%%%%%%%%%%%%%%%%%%%%%
\title{Developing back-end capabilities in Cyclus, a fuel cycle simulator}
\author{Gwendolyn J. Chee and Kathryn D. Huff}

\institute{
Dept. of Nuclear, Plasma and Radiological Engineering, University of Illinois at Urbana-Champaign \\
gchee2@illinois.edu
}

%%%% packages and definitions (optional)
\usepackage{graphicx} % allows inclusion of graphics
\usepackage{booktabs} % nice rules (thick lines) for tables
\usepackage{microtype} % improves typography for PDF
\usepackage{xspace}
\usepackage{tabularx}
\usepackage{subcaption}
\newcommand{\SN}{S$_N$}
\renewcommand{\vec}[1]{\bm{#1}} %vector is bold italic
\newcommand{\vd}{\bm{\cdot}} % slightly bold vector dot
\newcommand{\grad}{\vec{\nabla}} % gradient
\newcommand{\ud}{\mathop{}\!\mathrm{d}} % upright derivative symbol
\newcommand{\Cyclus}{\textsc{Cyclus}\xspace}%
\newcommand{\Cycamore}{\textsc{Cycamore}\xspace}%
\newcolumntype{c}{>{\hsize=.56\hsize}X}
\newcolumntype{b}{>{\hsize=.7\hsize}X}
\newcolumntype{s}{>{\hsize=.74\hsize}X}
\newcolumntype{f}{>{\hsize=.1\hsize}X}
\newcolumntype{a}{>{\hsize=.45\hsize}X}
\usepackage{titlesec}
\titleformat*{\subsection}{\normalfont}

\begin{document}
%%%%%%%%%%%%%%%%%%%%%%%%%%%%%%%%%%%%%%%%%%%%%%%%%%%%%%%%%%%%%%%%%%%%%%%%%%%%%%%%
\section{Abstract} 
Implementation of a nuclear waste disposal plan and minimizing the cost of the 
nuclear fuel cycle are crucial to the future use of nuclear power 
\cite{massachusetts_institute_of_technology_future_2003}. 
If the U.S. nuclear industry does not find an effective and safe plan to manage 
the waste, the nuclear industry will continue facing political and social 
opposition. 
It has been shown that permanent underground disposal of nuclear waste provides
excellent isolation from the human-inhabited environment 
\cite{rechard_evolution_2014}. 
Therefore, this work relies on the expectation that the chosen method of long 
term disposal of spent nuclear fuel (SNF) will be a deep geologic repository. 

Previous work towards the wicked problem of getting spent nuclear fuel from reactor 
sites to a final waste repository focuses on how different waste acceptance strategies 
impact economic expenduture \cite{nesbit_proposed_2015}, pre-emplacement 
surface storage time, waste package size, and repository 
footprint \cite{greenberg_application_2012}. 
There has also been efforts to holistically evaluate the entire system to consolidate 
how each factor impact the cost and safety of moving SNF from 
reactor sites to the final waste repository \cite{nutt_waste_2015}.
Previous work in studying repository loading have used spent fuel assemblies 
that have an average burn up composition \cite{johnson_optimizing_2016} 
to evaluate the heat load in the repository \cite{greenberg_application_2012}. 

Therefore, instead of using average SNF composition, this work aims to use U.S. 
historical SNF inventory data \cite{peterson_unf_standards_2017} in various 
simulations to study how waste acceptance strategies impact repository loading. 
This research aims to contribute realistic modeling tools for simulating how 
waste acceptance strategies impact the repository loading segment of the 
system since most of the current research is focused on the economic and 
transportation components of the problem. 

These simulations will be performed using \textsc{Cyclus}, an 
\textit{agent-based} fuel cycle simulation framework. 
In \textsc{Cyclus}, each facility in the fuel cycle is modeled individually 
and they interact with one another as independent \textit{agents}. 
The goal of this work is to develop a waste conditioning \textit{agent} 
and a waste repository \textit{agent} to provide \textsc{Cyclus} with 
the capabilities to run these simulations. 
Previously, a repository agent was created \cite{huff_cyclus_2013}, however, 
it is no longer compatible with the current \textsc{Cyclus} and 
did not have the capabilities to run these proposed simulations.  
 
In a waste repository, increasing spacing between waste canisters will reduce 
the risk of exceeding the thermal limit, however, by doing so a larger 
repository must be commissioned or less nuclear waste can be stored in a 
specific repository size. 
Therefore, the repository facility designed emplaces waste canisters by 
maximizing mass loading while remaining below the thermal limit of the 
host geologic media, with the purpose of minimizing repository size. 
The conditioning facility packages spent fuel assemblies into a waste canister 
that has user defined properties such as radius length and material thermal 
conductivity. 






%%%%%%%%%%%%%%%%%%%%%%%%%%%%%%%%%%%%%%%%%%%%%%%%%%%%%%%%%%%%%%%%%%%%%%%%%%%%%%%%
\bibliographystyle{ans}
\bibliography{bibliography}
\end{document}

