\documentclass{anstrans}
%%%%%%%%%%%%%%%%%%%%%%%%%%%%%%%%%%%
\title{Developing back-end capabilities in Cyclus, a fuel cycle simulator}
\author{Gwendolyn J. Chee and Kathryn D. Huff}

\institute{
Dept. of Nuclear, Plasma and Radiological Engineering, University of Illinois at Urbana-Champaign \\
gchee2@illinois.edu
}

%%%% packages and definitions (optional)
\usepackage{graphicx} % allows inclusion of graphics
\usepackage{booktabs} % nice rules (thick lines) for tables
\usepackage{microtype} % improves typography for PDF
\usepackage{xspace}
\usepackage{tabularx}
\usepackage{subcaption}
\newcommand{\SN}{S$_N$}
\renewcommand{\vec}[1]{\bm{#1}} %vector is bold italic
\newcommand{\vd}{\bm{\cdot}} % slightly bold vector dot
\newcommand{\grad}{\vec{\nabla}} % gradient
\newcommand{\ud}{\mathop{}\!\mathrm{d}} % upright derivative symbol
\newcommand{\Cyclus}{\textsc{Cyclus}\xspace}%
\newcommand{\Cycamore}{\textsc{Cycamore}\xspace}%
\newcolumntype{c}{>{\hsize=.56\hsize}X}
\newcolumntype{b}{>{\hsize=.7\hsize}X}
\newcolumntype{s}{>{\hsize=.74\hsize}X}
\newcolumntype{f}{>{\hsize=.1\hsize}X}
\newcolumntype{a}{>{\hsize=.45\hsize}X}
\usepackage{titlesec}
\titleformat*{\subsection}{\normalfont}

\begin{document}
%%%%%%%%%%%%%%%%%%%%%%%%%%%%%%%%%%%%%%%%%%%%%%%%%%%%%%%%%%%%%%%%%%%%%%%%%%%%%%%%
\section{Abstract}
Implementation of a nuclear waste disposal plan and the economics of the nuclear 
cycle are the largest obstacles to determining the future of nuclear power 
\cite{massachusetts_institute_of_technology_future_2003}. 
It has been shown that permanent underground disposal of nuclear waste provides 
excellent isolation from the human-inhabited environment 
\cite{rechard_evolution_2014}. 
Therefore, this work relies on the expectation that the chosen method of long 
term disposal of spent nuclear fuel (SNF) will be a deep geologic repository 
and that a site will eventually be selected.

The long term objective of this work is to study various nuclear waste repository 
loading strategies to determine which strategy is most effective at maximizing 
mass loading while remaining below a defined thermal limit for various host 
geologies. 

When approaching the wicked problem of getting spent nuclear fuel from the reactor 
sites to the final waste repository, previous work has approached it from various 
angles. 
Most of this work focuses on various waste acceptance strategies and how it 
impacts a list of different factors. 
These factors are interdependent. 
One factor is minimizing economic expenditure in terms minimizing the number 
of years SNF is left on post shutdown reactor sites by using a strategy of 
prioritizing fuel removal by longest shutdown plant first\cite{nesbit_proposed_2015}. 
Another factor is minimizing heat load concentration in the waste repository by 
using a mixed integer program \cite{johnson_optimizing_2016}. Other factors 
include pre-emplacement surface storage time, waste package size, and repository 
footprint where a sensitivity analysis was conducted to determine the trade-offs 
between them \cite{greenberg_application_2012}. 
A holistic evaluation of the entire system is important to determine the best 
way to implement the operation of moving SNF from reactor sites to the final 
waste repository. 
WMSA is looking at the problem from a overall viewpoint, trying to consolidate 
how each factor impacts the cost and safety of the final operation 
\cite{nutt_waste_2015}. 

In this work, the scope is limited to repository loading. 
Based on similar strategies studied above such as first-in-first-out, 
last-in-first-out fuel, combination allocation strategies, this work focuses
 on how the strategies impact the repository size and spacings while keeping 
 below the thermal limit. 
 This research contributes to the knowledge of how the waste acceptance 
 strategies will impact the repository loading component of the system since 
 most of the current research is focused on the economic and transportation 
 components of the problem. 

These simulations will be performed using \textsc{Cyclus}, an 
\textit{agent-based} fuel cycle simulation framework. 
In \textsc{Cyclus}, each facility in the fuel cycle is modeled individually 
and they interact with one another as independent \textit{agents}. 
The goal of this work is to develop a waste conditioning \textit{agent} 
and a waste repository \textit{agent} to provide \textsc{Cyclus} with 
the capabilities to run these simulations. 

For the task of loading of a waste repository, this work implements two 
constraints: keeping under the thermal limit and minimizing repository size. 
Increasing spacing between waste canisters in a waste repository will reduce 
the risk of exceeding the thermal limit, however, a larger repository must be 
commissioned or less nuclear waste can be stored in a specific repository size. 
Therefore, the repository facility designed emplaces waste canisters by 
maximizing mass loading while remaining below the thermal limit of the 
host geologic media. The conditioning facility packages spent fuel assemblies 
into a waste canister that has user defined properties such as radius length 
and material thermal conductivity. 


%%%%%%%%%%%%%%%%%%%%%%%%%%%%%%%%%%%%%%%%%%%%%%%%%%%%%%%%%%%%%%%%%%%%%%%%%%%%%%%%
\bibliographystyle{ans}
\bibliography{bibliography}
\end{document}

