\documentclass[11pt, letterpaper]{article}
\usepackage[utf8]{inputenc}
\usepackage{caption} % for table captions
\usepackage{amsmath} % for multi - line equations and piecewises
\usepackage{graphicx}
%\usepackage{textcomp}
\usepackage{xspace}
\usepackage{verbatim} % for block comments
%\usepackage{subfig} % for subfigures
\usepackage{enumitem} % for a) b) c) lists
\newcommand{\Cyclus}{\textsc{Cyclus}\xspace} %
\newcommand{\Cycamore}{\textsc{Cycamore}\xspace} %
\usepackage{tabularx}
\usepackage{color}
\usepackage{setspace}
\definecolor{bg}{rgb}{0.95, 0.95, 0.95}
\newcolumntype{b}{X}
\newcolumntype{f}{ > {\hsize=.15\hsize}X}
\newcolumntype{s}{ > {\hsize=.5\hsize}X}
\newcolumntype{m}{ > {\hsize=.75\hsize}X}
\newcolumntype{r}{ > {\hsize=1.1\hsize}X}
\usepackage{titling}
\usepackage[hang, flushmargin]{footmisc}
\renewcommand *\footnoterule{}
\graphicspath{{images /}}

\title{Developing back-end capabilities in Cyclus, a fuel cycle simulator
        \\ \vspace{0.5em} IHLRWM Abstract}
\author{Gwendolyn J. Chee}


\begin{document}
	\maketitle
	\hrule

\section * {}
\doublespacing
The largest barriers facing nuclear waste management in the U.S. 
are the political and social obstacles towards siting a final 
waste repository. 
It has been shown that permanent underground disposal of nuclear 
waste provides excellent isolation from the human-inhabited 
environment \cite{rechard_evolution_2014}. 
Therefore, this work relies on the expectation that the chosen 
method of long term disposal of spent nuclear fuel (SNF) will be 
a deep geologic repository and that a site will eventually be 
selected.
The long-term objective is to use U.S. historical SNF inventory data 
\cite{peterson_unf-st&dards_2017} in various simulations that 
model different transfer and loading strategies for moving SNF 
from reactor sites to a final waste repository. 
First-in-first-out and last-in-first-out fuel allocation 
strategies will be considered. 
These simulations will be performed using \textsc{Cyclus}, an 
\textit{agent-based} fuel cycle simulation framework. In 
\textsc{Cyclus}, each facility in the fuel cycle is modeled 
individually and they interact with one another as independent 
\textit{agents}. 
Therefore, the goal of this work is to develop a waste 
conditioning \textit{agent} and a waste repository \textit{agent} 
to provide \textsc{Cyclus} with the capabilities to run these 
simulations. 

The conditioning facility packages spent fuel assemblies into a 
waste canister that has user defined properties such as radius 
length and material thermal conductivity. 
For the task of loading of a waste repository, there are two
constraints: keeping under the thermal limit and 
minimizing repository size.
Increasing spacing between waste canisters in a waste repository
will reduce the risk of exceeding the thermal limit, however, 
a larger repository must be commissioned or less nuclear waste 
can be stored in a specific repository size. 
Therefore, the repository facility designed emplaces waste 
canisters by maximizing mass loading while remaining below 
the thermal limit of the host geologic media. 




\bibliographystyle{unsrt}
\bibliography{bibliography.bib}

\end{document}
