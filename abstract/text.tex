\documentclass{anstrans}
    %%%%%%%%%%%%%%%%%%%%%%%%%%%%%%%%%%%
    \title{Developing back-end capabilities in Cyclus, a fuel cycle simulator}
    \author{Gwendolyn J. Chee and Kathryn D. Huff}
    
    \institute{
    Dept. of Nuclear, Plasma and Radiological Engineering, University of Illinois at Urbana-Champaign \\
    gchee2@illinois.edu
    }
    
    %%%% packages and definitions (optional)
    \usepackage{graphicx} % allows inclusion of graphics
    \usepackage{booktabs} % nice rules (thick lines) for tables
    \usepackage{microtype} % improves typography for PDF
    \usepackage{xspace}
    \usepackage{tabularx}
    \usepackage{subcaption}
    \newcommand{\SN}{S$_N$}
    \renewcommand{\vec}[1]{\bm{#1}} %vector is bold italic
    \newcommand{\vd}{\bm{\cdot}} % slightly bold vector dot
    \newcommand{\grad}{\vec{\nabla}} % gradient
    \newcommand{\ud}{\mathop{}\!\mathrm{d}} % upright derivative symbol
    \newcommand{\Cyclus}{\textsc{Cyclus}\xspace}%
    \newcommand{\Cycamore}{\textsc{Cycamore}\xspace}%
    \newcolumntype{c}{>{\hsize=.56\hsize}X}
    \newcolumntype{b}{>{\hsize=.7\hsize}X}
    \newcolumntype{s}{>{\hsize=.74\hsize}X}
    \newcolumntype{f}{>{\hsize=.1\hsize}X}
    \newcolumntype{a}{>{\hsize=.45\hsize}X}
    \usepackage{titlesec}
    \titleformat*{\subsection}{\normalfont}
    
    \begin{document}
    %%%%%%%%%%%%%%%%%%%%%%%%%%%%%%%%%%%%%%%%%%%%%%%%%%%%%%%%%%%%%%%%%%%%%%%%%%%%%%%%
    \section{Abstract}
    Implementation of a nuclear waste disposal plan and minimizing the cost of the 
    nuclear fuel cycle are crucial to the future use of nuclear power 
    \cite{massachusetts_institute_of_technology_future_2003}. 
    If the U.S. nuclear industry does not find an effective and safe plan to manage 
    the waste, the nuclear industry will continue facing political and social 
    opposition. 
    Therefore, this work relies on the expectation that a deep geologic repository 
    for the disposal of spent nuclear fuel (SNF) will eventually be selected. 
    The objective of this work is to look at various nuclear waste repository loading 
    strategies to determine which strategy is most effective at maximizing 
    mass loading in a repository while remaining below a safe thermal limit 
    for various host geologies. 
    
    Previous work towards the wicked problem of getting spent nuclear fuel from reactor 
    sites to a final waste repository focuses on various waste acceptance strategies 
    and how it impacts a list of interdependent factors. 
    Nesbit et al found that economic expenditure is minimized by reducing the 
    number of years spent nuclear fuel (SNF) is left at post shutdown reactor 
    sites by using a strategy of prioritizing fuel removal by longest shutdown plant 
    first \cite{nesbit_proposed_2015}. 
    Johnson et al used a mixed integer program to minimize heat load concentration 
    in the waste repository \cite{johnson_optimizing_2016}. 
    Greenberg et al. conducted a sensitivity analysis to determine the trade-offs 
    between pre-emplacement surface storage time, waste package size, and repository 
    footprint \cite{greenberg_application_2012}. 
    WMSA is conducting a holistic evaluation of the entire system to consolidate 
    how each factor impacts the cost and safety, to determine the best 
    way to implement the operation of moving SNF from reactor sites to the final 
    waste repository\cite{nutt_waste_2015}. 
    
    The scope of this work is limited to how waste acceptance strategies impact 
    repository loading. 
    Waste acceptance strategies used in the abovementioned previous work such as 
    first-in-first-out, last-in-first-out fuel and other combination allocation 
    strategies are used in simulations to determine how these strategies impact 
    the repository size and spacings. 
    This research contributes to the knowledge of how the waste acceptance 
    strategies will impact the repository loading segment of the system since 
    most of the current research is focused on the economic and transportation 
    components of the problem. 
    
    These simulations will be performed using \textsc{Cyclus}, an 
    \textit{agent-based} fuel cycle simulation framework. 
    In \textsc{Cyclus}, each facility in the fuel cycle is modeled individually 
    and they interact with one another as independent \textit{agents}. 
    The goal of this work is to develop a waste conditioning \textit{agent} 
    and a waste repository \textit{agent} to provide \textsc{Cyclus} with 
    the capabilities to run these simulations. 
    
    Previous work in studying repository loading have used spent fuel assemblies 
    that have an average burn up composition. However, by using \textsc{Cyclus}, 
    this work uses U.S. historical SNF inventory data \cite{peterson_unf_standards_2017}
    in various simulations that model different transfer and loading strategies.  
    
    For loading of a waste repository, this work implements two 
    constraints: keeping under the thermal limit and minimizing repository size. 
    Increasing spacing between waste canisters in a waste repository will reduce 
    the risk of exceeding the thermal limit, however, a larger repository must be 
    commissioned or less nuclear waste can be stored in a specific repository size. 
    Therefore, the repository facility designed emplaces waste canisters by 
    maximizing mass loading while remaining below the thermal limit of the 
    host geologic media. The conditioning facility packages spent fuel assemblies 
    into a waste canister that has user defined properties such as radius length 
    and material thermal conductivity. 
    
    
    %%%%%%%%%%%%%%%%%%%%%%%%%%%%%%%%%%%%%%%%%%%%%%%%%%%%%%%%%%%%%%%%%%%%%%%%%%%%%%%%
    \bibliographystyle{ans}
    \bibliography{bibliography}
    \end{document}
    
    
